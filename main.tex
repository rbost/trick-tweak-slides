%!TEX spellcheck = en_US
%!TEX TS-options = -shell-escape

\documentclass{beamer}
\usepackage[utf8]{inputenc}
\usetheme{Madrid}
\useoutertheme{infolines}

\usecolortheme{seahorse}

%!TEX root = main.tex

% Use utf-8 encoding for foreign characters
\usepackage[utf8]{inputenc}
\usepackage[T1]{fontenc}

\usepackage{amsmath,amsfonts,amssymb}
\usepackage{amsthm}
% \usepackage{times}
% \usepackage{enumitem, framed}
\usepackage{color}
% \usepackage{tikz,tabularx}
\usepackage{graphicx}
\usepackage{stmaryrd}
% \usepackage{mathtools}
% \usepackage{epstopdf}

% \usepackage[shell]{gnuplottex}

% For protocols typesetting
\usepackage{algorithm}
\usepackage{algorithmicx}
\usepackage[]{algpseudocode}
\newcommand{\algsep}{\begin{center}\rule{\textwidth}{0.4pt}\end{center}}

\usepackage{fancybox}

\usepackage{float}

\newfloat{protocol}{tbph}{lop}
\floatname{protocol}{Protocol}

% New 'keywords' for algorithmicx
\newcommand{\InputA}[1]{\State \textbf{Input A:} #1}
\newcommand{\InputB}[1]{\State \textbf{Input B:} #1}
\newcommand{\OutputA}[1]{\State \textbf{Output A:} #1}
\newcommand{\OutputB}[1]{\State \textbf{Output B:} #1}
\algnewcommand\algorithmicto{\textbf{to}}
\algrenewtext{For}[3]%
  {\algorithmicfor\ $#1 = #2$ \algorithmicto\ $#3$ \algorithmicdo}

\algblockdefx[IFRED]{IfRed}{EndIfRed}
	[1]{\red{\algorithmicif\ {#1}}}
   	[0]{\red{\algorithmicend\ \algorithmicif}}

\algcblockx{IFRED}{ElseRed}{EndIfRed}
	[0]{\red{\algorithmicelse\ }}
	[0]{\red{\algorithmicend\ \algorithmicif}}
   
\algblockdefx[FORALLRED]{ForAllRed}{EndForRed}
   [1]{\red{\algorithmicforall\ {#1}}}
   [0]{\red{\algorithmicend\ \algorithmicfor}}

\algloopdefx[SMALLIF]{SmallIf}
	[1]{\algorithmicif\ {#1}}   
   
\algloopdefx[SMALLIFRED]{SmallIfRed}
	[1]{\red{\algorithmicif\ {#1}}}  
	 
\newcommand{\IfLine}[1]{\State \algorithmicif\ {#1}}   
\newcommand{\ElseLine}{\State \algorithmicelse\ }   
\newcommand{\IfLineRed}[1]{\State \red{\algorithmicif\ {#1}}}   
  
\algdef{SE}[DOWHILE]{Do}{doWhile}{\algorithmicdo}[1]{\algorithmicwhile\ #1}%

\algrenewcommand\algorithmicindent{1.0em}%
\algrenewcommand\alglinenumber[1]{\scriptsize #1:}
\algrenewcommand{\algorithmiccomment}[1]{\hfill\hspace{.2cm} $\triangleright$\ \textit{#1}}



% requires amsthm, enumitem
% \theoremstyle{algorithms}
% \newtheorem{construction}{Algorithm}
% \newcommand{\ALGORITHM}[4]{%  name, llabel, intro, \items
%   \begin{construction}[#1]\label{#2} \mbox{}
%   \normalfont\noindent
%   #3
%   \vspace{0.13\baselineskip}
%   \begin{enumerate}[noitemsep,nolistsep]\itemsep=0.1\baselineskip
%   #4
%   \end{enumerate}
%   \end{construction}
% }

%%%%%%%%%%%%  Defining theorem-like environments %%%%%%%%%%

% \ifnotvldb

\usepackage{amsthm}
\newtheorem*{rep@theorem}{\rep@title}
\newcommand{\newreptheorem}[2]{%
\newenvironment{rep#1}[1]{%
 \def\rep@title{#2 \ref{##1}}%
 \begin{rep@theorem}}%
 {\end{rep@theorem}}}
\makeatother
% \fi

% \newtheorem{theorem}{Theorem}

% \ifnotvldb
\newreptheorem{theorem}{Theorem}
% \fi

\newtheorem{proposition}[theorem]{Proposition}
% \newtheorem{informaltheorem}[theorem]{Theorem}
% \newtheorem{informalcorollary}[theorem]{Corollary}
% \newtheorem{informalclaim}[theorem]{Claim}
% %\newtheorem{definition}{Definition}[section]
% \ifvldb
% \newdef{definition}{Definition}
% \else
% \newtheorem{definition}{Definition}
% \fi
% \newtheorem{claim}[theorem]{Claim}
% \newtheorem{remark}{Remark}[section]
% \newtheorem{remark}[theorem]{Remark}
% \newtheorem{lemma}[theorem]{Lemma}
% \newtheorem{corollary}[theorem]{Corollary}
% \newtheorem{fact}[theorem]{Fact}
% \newtheorem{assumption}{Assumption}

\frenchspacing

\usepackage{latexsym}
\usepackage{color}
\usepackage{url,array,epsfig}
% \usepackage[pageanchor=false,colorlinks=true,citecolor=blue,linkcolor=blue,urlcolor=black]{hyperref}

\newcommand{\remove}[1]{}
\newcommand{\ignore}[1]{}

%% For gorgeous subfigures
\usepackage{caption}
\captionsetup{justification=justified,labelfont=bf,labelsep=endash}
% \usepackage{subcaption}
\usepackage{dblfloatfix}                 

% \setlength{\belowcaptionskip}{-.5cm}


\usepackage{multicol}

%% for the code %%
\usepackage{listings}
\lstset{language=c++}
\lstdefinestyle{customc}{
  belowcaptionskip=1\baselineskip,
  breaklines=true,
  % frame=L,
  xleftmargin=\parindent,
  language=C,
  showstringspaces=false,
  basicstyle=\footnotesize\ttfamily,
  keywordstyle=\bfseries\color{green!40!black},
  commentstyle=\itshape\color{purple!40!black},
  identifierstyle=\color{blue},
  stringstyle=\color{orange},
}
\lstset{escapechar=@,style=customc}

%% For footnotes in tables
% \usepackage{footnote}
% \makesavenoteenv{tabular}
% \makesavenoteenv{table}
% \makesavenoteenv{table*}
\usepackage{multirow}

\newcommand{\red}[1]{\textcolor{red}{#1}}

\usepackage{tikz}
\usetikzlibrary{shapes,positioning,calc}
\usetikzlibrary{arrows}
\usetikzlibrary{fit}

\usepackage{gnuplot_sty/gnuplot-lua-tikz}

\usepackage{boxedminipage}



\usepackage{pifont}
\newcommand{\cmark}{\ding{51}}%
\newcommand{\xmark}{\ding{55}}%


%Security Parameter
\newcommand{\secparam}{\kappa}
\newcommand{\secp}{\secparam}


% Emphasis in theorems
\newcommand{\emphth}[1]{{\rm\sf {#1}}}

% Changing QED symbol in claim proofs
%\newenvironment{claimproof}{\begin{IEEEproof}
%\renewcommand{\qedsymbol}{{$\blacksquare$}}
%}{\end{IEEEproof}}


% Fixing bug in citations inside of brackets.
\newcommand{\brafix}[1]{[{#1}]}
\newcommand{\citefix}[2]{{\cite[#1]{#2}}}

% Math operators (from Raphael)
\DeclareMathOperator*{\argmax}{argmax}
\newcommand{\llb}{\llbracket}
\newcommand{\rrb}{\rrbracket}
\newcommand{\bb}[1]{\llbracket #1 \rrbracket}
\newcommand{\bbb}[1]{[\hspace{-1.5pt}\llbracket #1 \rrbracket\hspace{-1.5pt}]}


%%%%%%%%%%%%%%%%%%%%%%%%%%%%%%%%%%%%%%%%%%%%%%%%%%%%%%%%%%%%

% Calligraphic shortcuts.

\def\cA{{\cal A}}
\def\cB{{\cal B}}
\def\cC{{\cal C}}
\def\cD{{\cal D}}
\def\cE{{\cal E}}
\def\cF{{\cal F}}
\def\cG{{\cal G}}
\def\cH{{\cal H}}
\def\cI{{\cal I}}
\def\cJ{{\cal J}}
\def\cK{{\cal K}}
\def\cL{{\cal L}}
\def\cM{{\cal M}}
\def\cN{{\cal N}}
\def\cO{{\cal O}}
\def\cP{{\cal P}}
\def\cQ{{\cal Q}}
\def\cR{{\cal R}}
\def\cS{{\cal S}}
\def\cT{{\cal T}}
\def\cU{{\cal U}}
\def\cV{{\cal V}}
\def\cW{{\cal W}}
\def\cX{{\cal X}}
\def\cY{{\cal Y}}
\def\cZ{{\cal Z}}
%%%%%%%%%%%%%%%%%

\def\veca{\vc{a}}
\def\vecb{\vc{b}}
\def\vecc{\vc{c}}
\def\vecd{\vc{d}}
\def\vece{\vc{e}}
\def\vecm{\vc{m}}
\def\vecr{\vc{r}}
\def\vecs{\vc{s}}
\def\vect{\vc{t}}
\def\vecv{\vc{v}}
\def\vecx{\vc{x}}
\def\vecy{\vc{y}}

\def\A{\mathcal{A}}
\def\B{\mathcal{B}}
\def\C{\mathcal{C}}
\def\D{\mathcal{D}}
\def\E{\mathcal{E}}
\def\I{\mathcal{I}}
\def\K{\mathcal{K}}
\def\M{\mathcal{M}}
\def\T{\mathcal{T}}
\def\U{\mathcal{U}}


\def\Z{\mathbb{Z}}
\def\N{\mathbb{N}}
\def\R{\mathbb{R}}
\def\Q{\mathbb{Q}}
\def\F{\mathbb{F}}
\def\P{\mathbb{P}}

\def\bbC{{\mathbb C}}
\def\bbE{{\mathbb E}}
\def\bbF{{\mathbb F}}
\def\bbG{{\mathbb G}}
\def\bbM{{\mathbb M}}
\def\bbN{{\mathbb N}}
\def\bbQ{{\mathbb Q}}
\def\bbR{{\mathbb R}}
\def\bbV{{\mathbb V}}
\def\bbZ{{\mathbb Z}}

\def\tE{\widetilde{E}}
\def\trho{\widetilde{\rho}}
\def\tpi{\widetilde{\pi}}


\def\Zq{\bbZ_q}
\def\Zp{\bbZ_p}


\newcommand{\hvcs}{\hat{\vc{s}}}
\newcommand{\hmxA}{\hat{\mx{A}}}
\newcommand{\hvce}{\hat{\vc{e}}}
\newcommand{\hvcb}{\hat{\vc{b}}}
\newcommand{\he}{{\hat{e}}}
\newcommand{\heta}{{\hat{\eta}}}
\newcommand{\hpsi}{{\hat{\psi}}}
\newcommand{\hPsi}{{\hat{\Psi}}}
\newcommand{\hchi}{{\hat{\chi}}}
\newcommand{\hvcv}{{\hat{\vc{v}}}}
\newcommand{\hw}{{\hat{w}}}
\newcommand{\hvca}{{\hat{\vc{a}}}}
\newcommand{\hb}{{\hat{b}}}
\newcommand{\hB}{{\hat{B}}}
\newcommand{\homega}{{\hat{\omega}}}

\newcommand{\eps}{\varepsilon}

% enclosers

\newcommand{\ceil}[1]{\left\lceil #1 \right\rceil}
\newcommand{\floor}[1]{\left\lfloor #1 \right\rfloor}
\newcommand{\round}[1]{\left\lfloor #1 \right\rceil}
\newcommand{\ip}[1]{\left\langle #1 \right\rangle}



% Other short-hands


\def\pmset{\{\pm 1\}}
\def\ind{\mathbbm{1}}
%\def\ind{\mathbf{1}}

\newcommand{\abs}[1]{\left\vert {#1} \right\vert}
\newcommand{\norm}[1]{\left\| {#1} \right\|}
\newcommand{\twonorm}[1]{\left\| {#1} \right\|_2}
\newcommand{\norminf}[1]{\left\| {#1} \right\|_{\infty}}


% Assignments
\def\getsr{\stackrel{\scriptscriptstyle{\$}}{\gets}}
\def\getsd{{:=}}
%\def\bydef{\stackrel{.}{=}}
\def\bydef{\triangleq}
\def\getsf{{\gets}}
\newcommand{\rand}{\getsr}


\newcommand{\leftprotarrow}[2]{\xleftarrow{\makebox[#1]{#2}}}
\newcommand{\rightprotarrow}[2]{\xrightarrow{\makebox[#1]{#2}}}
\newcommand{\protarrow}[2]{\xleftrightarrow{\makebox[#1]{#2}}}
\newcommand{\getshook}{\scalebox{1.2}{\rotatebox[origin=c]{90}{$\Rsh$}}\ }


% Asymptotics

\def\poly{{\rm poly}}
\def\polylog{{\rm polylog}}
\def\polyloglog{{\rm polyloglog}}
\def\negl{{\rm negl}}
\newcommand{\ppt}{\mbox{{\sc ppt}}}
\def\Otilde{\widetilde{O}}

% Indistinguishability
\newcommand{\cind}{{\ \stackrel{c}{\approx}\ }}
\newcommand{\compind}{\cind}
\newcommand{\sind}{{\ \stackrel{s}{\approx}\ }}

%\newcommand{\sind}[1][\epsilon]{{\ \stackrel{{#1}}{\approx}_{s}\ }}
%\newcommand{\epssind}{{\ \stackrel{{\epsilon}}{\approx}_{s}\ }}


% Complexity classes

\def\NP{\mathbf{NP}}
\def\Ppoly{{\mathbf{P}/\poly}}

% Advantage
\newcommand{\Adv}{\mathtt{Adv}}


% Encryption & Co.
\newcommand{\Enc}{\mathsf{Enc}}
\newcommand{\Dec}{\mathsf{Dec}}

% Cryptographic assumptions

\newcommand{\ddh}{\mathrm{DDH}}
\newcommand{\cdh}{\mathrm{CDH}}
\newcommand{\dlin}{\text{\rm $d$LIN}}
\newcommand{\lin}{\text{\rm Lin}}
\newcommand{\sxdh}{\mathrm{SXDH}}
\newcommand{\rsa}{\mathrm{RSA}}
\newcommand{\sis}{\mathrm{SIS}}
\newcommand{\XPis}{\mathrm{ISIS}}
\newcommand{\lwe}{\mathsf{LWE}}
\newcommand{\qr}{\mathrm{QR}}


\newcommand{\mx}[1]{\mathbf{{#1}}}
% \newcommand{\vc}[1]{\mathbf{{#1}}}
\newcommand{\vc}[1]{\vec{{#1}}}
\newcommand{\gvc}[1]{\bm{{#1}}}
%\newcommand{\vc}[1]{\gvc{{#1}}}

\newcommand{\zo}{\{0,1\}}

% Probability

\newcommand{\Ex}{\mathop{\bbE}}
\newcommand{\cov}{\mathop{\text{\rm Cov}}}

%\newcommand{\sd}{\mathop{\text{\tt dist}}}
\newcommand{\sd}{\mathop{\Delta}}



% Cryptographic elements

\newcommand{\evk}{\mathsf{evk}}
\newcommand{\pp}{{pp}}
\newcommand{\vk}{{vk}}
\newcommand{\td}{{td}}

\def\PK{\mathsf{PK}}
\def\MPK{\mathsf{MPK}}
\def\MSK{\mathsf{MSK}}
\def\SK{\mathsf{SK}}
\def\bydef{:=}

\newcommand{\prf}{{{\mathtt{prf}}}}
\newcommand{\prp}{{{\mathtt{prp}}}}
\newcommand{\tprf}{{\widetilde{\mathtt{prf}}}}
\newcommand{\tprp}{{\widetilde{\mathtt{prp}}}}

\newcommand{\zeon}{\{0, 1\}}

% Primitives
\usepackage[colorinlistoftodos]{todonotes}

% \renewcommand{\max}{\mathsf{max}}

\newcommand{\gf}{{\text{GF}}}

\newcommand{\zset}[1]{\{0, \ldots, {#1}\}}
\newcommand{\oset}[1]{\{1, \ldots, {#1}\}}


\newcommand{\tab}{\ \ \ }
\newcommand{\ttab}{\ \ \ \ \ \ }
\newcommand{\tttab}{\ \ \ \ \ \ \ \ \ }


\newcommand{\sk}{\mathsf{sk}}
\newcommand{\pk}{\mathsf{pk}}


\newcommand{\xref}[1]{Section~\ref{#1}}
\newcommand{\fig}[1]{Figure~\ref{#1}}

% Compact itemize and enumerate.  Note that they use the same counters and
% symbols as the usual itemize and enumerate environments.
\makeatletter
\def\compactify{\itemsep=3pt plus3pt \topsep=3pt plus3pt \partopsep=0pt
\parsep=0pt \leftmargin=1.3em}
\let\latexusecounter=\usecounter
\def\CompactItemize{%
  \ifnum \@itemdepth >\thr@@\@toodeep\else
    \advance\@itemdepth\@ne
    \edef\@itemitem{labelitem\romannumeral\the\@itemdepth}%
    \expandafter
    \list
      \csname\@itemitem\endcsname
      {\compactify\def\makelabel##1{\hss\llap{##1}}}%
  \fi}
\let\endCompactItemize\endlist
\newenvironment{CompactEnumerate}
  {\def\usecounter{\compactify\latexusecounter}
   \begin{enumerate}}
  {\end{enumerate}\let\usecounter=\latexusecounter}
\makeatother


%!TEX root = main.tex

\newcommand{\DB}{\mathsf{DB}}
\newcommand{\HistDB}{\mathsf{HistDB}}
\newcommand{\Hist}{\mathsf{Hist}}
\newcommand{\UpHist}{\mathsf{UpHist}}
\newcommand{\EDB}{\mathsf{EDB}}
\newcommand{\SSE}{\mathsf{SSE}}
\renewcommand{\ind}{\mathsf{ind}}
\newcommand{\W}{\mathsf{W}}
\newcommand{\EDBSetup}{\mathsf{EDBSetup}}
\newcommand{\Setup}{\mathsf{Setup}}
\newcommand{\Search}{\mathsf{Search}}
\newcommand{\Retrieve}{\mathsf{Retrieve}}
\newcommand{\Update}{\mathsf{Update}}
\newcommand{\Verify}{\mathsf{Verify}}
\newcommand{\VerifySearch}{\Verify\Search}
\newcommand{\VerifyUpdate}{\Verify\Update}
\newcommand{\Challenge}{\mathsf{Challenge}}

\newcommand{\Lookup}{\mathsf{Lookup}}
\newcommand{\Query}{\mathsf{Query}}


\newcommand{\ACCEPT}{\mathtt{ACCEPT}}
\newcommand{\REJECT}{\mathtt{REJECT}}

\newcommand{\TSet}{\mathsf{TSet}}
\newcommand{\ESet}{\mathsf{ESet}}
\newcommand{\XTable}{\mathsf{XTable}}
\newcommand{\stag}{\mathsf{stag}}
\newcommand{\token}{\mathsf{token}}
\newcommand{\xtoken}{\mathsf{xtoken}}

\newcommand{\BF}{\mathsf{BF}}
\newcommand{\UpTable}{\mathsf{UpTable}}

\newcommand{\TSetSetup}{\mathsf{TSetSetup}}
\newcommand{\TSetGetTag}{\mathsf{TSetGetTag}}
\newcommand{\TSetRetrieve}{\mathsf{TSetRetrieve}}
\newcommand{\TSetDecrypt}{\mathsf{TSetDecrypt}}
\newcommand{\TSetSearch}{\mathsf{TSetSearch}}
\newcommand{\TSetUpdate}{\mathsf{TSetUpdate}}

\newcommand{\ESetSetup}{\mathsf{ESetSetup}}
\newcommand{\ESetGetToken}{\mathsf{ESetGetToken}}
\newcommand{\ESetRetrieve}{\mathsf{ESetRetrieve}}
\newcommand{\ESetDecrypt}{\mathsf{ESetDecrypt}}
\newcommand{\ESetUpdate}{\mathsf{ESetUpdate}}

\newcommand{\CounterSetup}{\mathsf{CounterSetup}}
\newcommand{\CounterToken}{\mathsf{CounterToken}}
\newcommand{\CounterGet}{\mathsf{CounterGet}}
\newcommand{\CounterVerify}{\mathsf{CounterVerify}}

\newcommand{\VHT}{\mathsf{VHT}}
\newcommand{\VHTSetup}{\mathsf{VHTSetup}}
\newcommand{\VHTGet}{\mathsf{VHTGet}}
\newcommand{\VHTVerify}{\mathsf{VHTVerify}}
\newcommand{\VHTUpdate}{\mathsf{VHTUpdate}}
\newcommand{\VHTRefresh}{\mathsf{VHTRefresh}}
\newcommand{\hkey}{\mathsf{hkey}}

\newcommand{\Find}{\mathsf{Find}}


\newcommand{\Init}{\mathsf{Init}}
\newcommand{\Final}{\mathsf{Final}}
\newcommand{\op}{\mathsf{op}}
\newcommand{\add}{\mathsf{add}}
\newcommand{\del}{\mathsf{del}}
\newcommand{\edit}{\mathsf{edit}}
\newcommand{\opin}{\mathsf{in}}
\newcommand{\data}{\mathsf{data}}

\newcommand{\KeyGen}{\mathsf{KeyGen}}

%% ORAM %%
\newcommand{\PosMap}{\mathsf{PosMap}}
\newcommand{\BucketMap}{\mathsf{BucketMap}}
\newcommand{\Stash}{\mathsf{Stash}}
\newcommand{\opread}{\mathsf{read}}
\newcommand{\opwrite}{\mathsf{write}}
\newcommand{\rnd}{\mathsf{round}}
\newcommand{\addr}{\mathsf{addr}}
% \newcommand{\block}{\mathsf{block}}

\newcommand{\EvictPath}{\textsc{EvictPath}}
\newcommand{\ReadBucket}{\textsc{ReadBucket}}
\newcommand{\WriteBucket}{\textsc{WriteBucket}}

\newcommand{\DirectORAM}{\textsc{DirectORAM}}

\newcommand{\qp}{\mathsf{qp}}
\renewcommand{\sp}{\mathsf{sp}}

\newcommand{\bad}{\mathsf{bad}}
\newcommand{\true}{\mathsf{true}}
\newcommand{\false}{\mathsf{false}}


\newcommand{\arrT}{\mathbf{T}}
\newcommand{\arrt}{\mathbf{t}}

\renewcommand{\L}{\mathcal{L}}

\newcommand{\vecq}{\mathbf{q}}
\newcommand{\vecw}{\mathbf{w}}


\newcommand{\SP}{\mathsf{SP}}
\newcommand{\RP}{\mathsf{RP}}
\newcommand{\IP}{\mathsf{IP}}
% \newcommand{\IS}{\mathsf{IS}}
\newcommand{\XP}{\mathsf{XP}}
\newcommand{\IE}{\mathsf{IE}}
\newcommand{\tIE}{\widetilde{\mathsf{IE}}}
\newcommand{\bIE}{\overline{\mathsf{IE}}}
\newcommand{\X}{\mathsf{XTokens}}

\newcommand{\parts}{\mathcal{P}}

\def\Adv{{\bf Adv}}
\def\AdvCor{{\bf AdvCor}}
\def\AdvConf{{\bf AdvConf}}
\def\AdvComp{{\bf AdvComp}}
\def\AdvSound{{\bf AdvSnd}}
\def\AdvColl{{\bf AdvColl}}

\newcommand{\FSMKSE}{\textbf{FSMKSE}}
\newcommand{\CXT}{\ensuremath{ \mathsf{CXT}}}
\newcommand{\CXXT}{\ensuremath{ \mathsf{CXXT}}}
\newcommand{\OXT}{\ensuremath{ \mathsf{OXT}}}
\newcommand{\GSV}{\ensuremath{ \mathsf{GSV}}}
\newcommand{\Real}{\mathbf{Real}}
\newcommand{\Ideal}{\mathbf{Ideal}}

\newcommand{\Perms}{\textsf{Perms}}
\newcommand{\WPerms}{\textsf{WPerms}}

\newcommand{\Chi}{\text{\large{$\chi$}}}
\newcommand{\lpi}{\text{\Large{$\pi$}}}

\newcommand{\SPS}{\mathsf{SPS}}
\newcommand{\VSPS}{{\mathsf{Verif}\text{-}\mathsf{SPS}}}
\newcommand{\MAC}{\mathsf{MAC}}
\newcommand{\AEnc}{\mathsf{AEnc}}
\newcommand{\ADec}{\mathsf{ADec}}

\newcommand{\AllHoles}{\mathsf{AllHoles}}
\newcommand{\VerifyHoles}{\mathsf{VerifyHoles}}
\newcommand{\CheckResults}{\mathsf{CheckResults}}
\newcommand{\ProveHoles}{\mathsf{ProveHoles}}
\newcommand{\Holes}{\mathsf{Holes}}
\newcommand{\ProcessLevel}{\mathsf{ProcessLevel}}

\def\prove{{\rm prf}}
\def\check{{\rm chk}}
\def\build{{\rm build}}

\newcommand{\Stp}{\mathsf{Stp}}
\newcommand{\Srch}{\mathsf{Srch}}
\newcommand{\Ret}{\mathsf{Ret}}
\newcommand{\Updt}{\mathsf{Updt}}

\def\gettrans{\stackrel{\scriptscriptstyle{\mathsf{trans}}}{\gets}}

\newcommand{\mask}{\mathtt{mask}}

\newcommand{\sophos}{{\Sigma o\phi o\varsigma}}

\newcommand{\comment}[1]{\textcolor{red}{\bf #1}}

\newcommand{\type}{\mathsf{type}}
\newcommand{\param}{\mathsf{param}}


\newcommand{\cDB}{\mathcal{DB}}
% \newcommand{\tf}{\widetilde{f}}
% \newcommand{\tbf}{\widetilde{\mathbf{f}}}
\newcommand{\be}{{\mathbf{e}}}
\newcommand{\te}{\widetilde{e}}

\DeclareMathOperator*{\var}{\mathbb{V}ar}


\title{Trick or Tweak}
\subtitle{On the (In)security of OTR's Tweaks}
\author{Raphael Bost\inst{1,2}\and Olivier Sanders\inst{3}}
% \institute[short]{\inst{1} Direction Générale de l'Armement - Maîtrise de l'Information \and \inst{2} Université de Rennes 1 \samelineand \inst{3} Orange Labs}
\newcommand{\samelineand}{\qquad}


% \institute[short]{
%   \begin{tabular}[h]{cc}
%       \begin{minipage}{5cm}
% 		  \inst{1} Direction Générale de l'Armement - Maîtrise de l'Information \\
% 	  \end{minipage}
% 			  &
% 	  \begin{minipage}{5cm}
% 		  \inst{3} Orange Labs
% 	  \end{minipage}\\
% 				  \inst{2} Université de Rennes 1
% 				  &
%   \end{tabular}
% }
	  \institute[]{
	    \inst{1} Direction Générale de l'Armement - Maîtrise de l'Information\\
	  	\inst{2} Université de Rennes 1
	  	\and
	  	\inst{3} Orange Labs
	  }
\date[AC '16]{Asiacrypt 2016, Hanoi}

\begin{document}

	\begin{frame}
		\titlepage
	\end{frame}

\section{Presentation of OTR} % (fold)
\label{sec:otr_presentation}

	\begin{frame}
		\frametitle{Offset Two Rounds (OTR)}
		\begin{itemize}
			\item CAESAR submission by K. Minematsu (Eurocrypt '14)
			\item Inspired by OCB
			\item Rate-1 AE
			\item Tweakable blockcipher based
			\item Inverse-free (only needs $E$, not $E^{-1}$)
			\item Two rounds Feistel construction
			\item Defined for any block size $n$.
		\end{itemize}
		
	\end{frame}

	\begin{frame}
		\frametitle{Tweakable Blockcipher (TBC) [LRW02]}
		Add a public input to a blockcipher -- the tweak -- to add variability.

		\vspace{0.4cm}

		\begin{block}{Block cipher (a.k.a PRP)}
			$E_K$ is indistinguishable from a random permutation $\pi$.
					\[
					\P[K \rand \K : \A^{E_K(.)} \Rightarrow 1] - \P[\pi \rand \text{Perm}(n) : \A^{\pi(.)} \Rightarrow 1] \leq \negl(\lambda)
					\]
		\end{block}
		
		\vspace{0.4cm}
		\begin{block}{Tweakable Blockcipher (a.k.a tweakable PRP)}
		The $T \in \mathcal{T}$ indexed permutation family $\tE_K(T,.)$ is indistinguishable from a random permutation family $\pi(T,.)$
		\[
		\P[K \rand \K : \A^{\tE_K(.,.)} \Rightarrow 1] - \P[\tpi \rand \text{Perm}(\mathcal{T},n) : \A^{\tpi(.,.)} \Rightarrow 1] \leq \negl(\lambda)
		\]
		\end{block}
		
	\end{frame}
	
	
\def\crossoffset{0.8cm}

	\begin{frame}
		\frametitle{OTR Encryption (1/2)}
				\begin{center}
					\begin{tikzpicture}[
						scale=1,
						node distance=1.8cm,
						line width = 0.5pt,
					]
						%!TEX root = ../main.tex
%!TEX root = ../main.tex

\tikzstyle{GW}=[draw,rectangle,minimum height=2em,inner sep=3pt,thin]
\tikzstyle{SW}=[draw,rectangle,minimum height=1.25em,inner sep=1pt,thin]
\tikzstyle{XOR}=[inner sep=0pt]

	% \begin{scope}
	% 	\node (N) {$N$};
	% 	\node[SW,below=0.4cm of N] (p) {pad};
	% 	\node[GW,below = 0.4cm of p] (e) {$E_{K}$};
	% 	\node[below = 0.4cm of e] (delta) {$\delta$};
	%   		\node[SW, below = 0.4cm of delta] (mult) {$\times x^2$};
	%   		\node[ below = 0.4cm of mult] (L) {$L$};
	%
	% 	\draw[->] (N) -- (p);
	% 	\draw[->] (p) -- (e);
	% 	\draw[->] (e) -- (delta);
	% 	\draw[->] (delta) -- (mult);
	% 	\draw[->] (mult) -- (L);
	%
	% \end{scope}

	\begin{scope}
		\node (m1) {$M[1]$};
		\node[ right=3cm of m1.west] (m2) {$M[2]$};

		\node[GW,below right=0.5cm and 1cm of m1.south] (e1) {$\tE^{N,1,0}_{K}$};	
		\node[XOR] at (e1 -| m2) (x1) {$\bigoplus$};
		\node[GW,below=1.3cm of e1] (e2) {$\tE^{N,1,1}_{K}$};	
		\node[XOR] at (e2 -| m2) (x2) {$\bigoplus$};
		\node[below=4cm of m1] (c1) {$C[1]$};
		\node[ below=4cm of m2] (c2) {$C[2]$};

		\draw[->]  (m1) |- (e1);
		\draw (m2) -- (x1);  	
		\draw (e1) -- (x1);
%
%		\draw (e2 -| m2) -- (m1 |- e1);
		\draw let \p3 = (e2 -| m2) in (m1 |- e1) -- ++(0,-\crossoffset) -- (\x3,\y3+\crossoffset)  -- (x2);
%		
		\draw let \p1 = (m1 |- e2),\p2 = (e2),\p3 = (x1) in (x1) -- ++(0,-\crossoffset) -- (\x1,\y1+\crossoffset)  -- (c1);
%  			
		\draw[->] (m1 |- e2) -- (e2);
		\draw (e2) -- (x2);
%
 		\draw (x2) -- (c2);  	

	\end{scope}
  
%
	\node at (4.3,-2.75) {$\dots\dots$};
%  

	\begin{scope}[xshift=5.5cm]
		\node (m1) {$M[2\ell-3]$};
		\node[ right=3cm of m1.west] (m2) {$M[2\ell-2]$};

		\node[GW,below right=0.5cm and 1cm of m1.south] (e1) {$\tE^{N,\ell-1,0}_{K}$};	
		\node[XOR] at (e1 -| m2) (x1) {$\bigoplus$};
		\node[GW,below=1.3cm of e1] (e2) {$\tE^{N,\ell-1,1}_{K}$};	
		\node[XOR] at (e2 -| m2) (x2) {$\bigoplus$};
		\node[below=4cm of m1] (c1) {$C[2\ell-3]$};
		\node[ below=4cm of m2] (c2) {$C[2\ell-2]$};

		\draw[->]  (m1) |- (e1);
		\draw (m2) -- (x1);  	
		\draw (e1) -- (x1);
%
		\draw let \p3 = (e2 -| m2) in (m1 |- e1) -- ++(0,-\crossoffset) -- (\x3,\y3+\crossoffset)  -- (x2);
%		
		\draw let \p1 = (m1 |- e2),\p2 = (e2),\p3 = (x1) in (x1) -- ++(0,-\crossoffset) -- (\x1,\y1+\crossoffset)  -- (c1);
%  			
		\draw[->] (m1 |- e2) -- (e2);
		\draw (e2) -- (x2);
%
 		\draw (x2) -- (c2);  	

	\end{scope}
 
					\end{tikzpicture}
				\end{center}		
	\end{frame}

	\begin{frame}
		\frametitle{OTR Encryption (2/2)}
				\begin{center}

						%!TEX root = main.tex
\tikzstyle{GW}=[draw,rectangle,minimum height=2em,inner sep=3pt,thin]
\tikzstyle{SW}=[draw,rectangle,minimum height=1.25em,inner sep=1pt,thin]
\tikzstyle{XOR}=[inner sep=0pt]
\tikzstyle{vert}=[inner sep=0pt,text width=1pt]
	


\begin{tabular}{ccc}
% \begin{tabular}{p{15em}p{15em}p{8em}}
\centering{if $m$ is even}  & \centering{if $m$ is odd} & Tag\\

\begin{tikzpicture}[
				scale=1,
				node distance=1.8cm,
				line width = 0.5pt,
			]
	
	\begin{scope}
		\node (m1) {$M[m-1]$};
		\node[ right=3.5cm of m1.west] (m2) {$M[m]$};
		\node[GW,below right=0.5cm and -0.4cm of m1] (e1) {$\tE^{N,\ell,0}_{K}$};	
		\node[SW,right=0.2cm of e1] (msb) {$\mathtt{msb}$};
		\node[XOR] at (e1 -| m2) (x1) {$\bigoplus$};
				
		\node[SW, below=1.5cm of e1.west] (p1) {$\mathtt{pad}$};
		\node[GW,right=0.5cm of p1] (e2) {$\tE^{N,\ell,1}_{K}$};	
		\node[XOR] at (e2 -| m2) (x2) {$\bigoplus$};
				
%
		\node[below=4cm of m1] (c1) {$C[m-1]$};
		\node[below=4cm of m2] (c2) {$C[m]$};
%
%		% \draw[edge] let \p1 = (m1),\p2 = (e1) in (m1) -- (\x1,\y2)  -- (e1);
		\draw[->]  (m1) |- (e1);
		\draw (m2) -- (x1);
		  \draw[->] (e1) -- (msb);
		\draw (msb) -- (x1);
%
%		\draw (e2 -| m2) -- (m1 |- e1);
		\draw let \p3 = (e2 -| m2) in (m1 |- e1) -- ++(0,-0.7*\crossoffset) -- (\x3,\y3+0.7*\crossoffset)  -- (x2);
%		
		\draw let \p1 = (m1 |- e2),\p2 = (e2),\p3 = (x1) in (x1) -- ++(0,-0.7*\crossoffset) -- (\x1,\y1+0.7*\crossoffset) -- (m1 |- e2);
%  			

		\draw[->] (m1 |- e2) -- (p1);
		\draw[->] (p1) -- (e2);
		\draw (e2) -- (x2);
%

		\draw let \p1 = (c1) in (x2) -- ++(0,-0.5cm)  -- (\x1,\y1+0.75cm) -- (c1);		
		\draw let \p1 = (c2),\p2 = (m1 |- e2),\p3 = (x2) in (m1 |- e2) -- ++(0,-0.5cm)  -- (\x1,\y1+0.75cm) -- (c2);

	\end{scope}

		\end{tikzpicture}
& 
\begin{tikzpicture}[
				scale=1,
				node distance=1.8cm,
				line width = 0.5pt,
			]
	
	\begin{scope}
		\node (m1) {$M[m]$};
		\node [left = 1cm of m1](zero) {$0^n$};
		\node[GW,below=1cm of zero] (e1) {$\tE^{N,\ell,1}_{K}$};	
		\node[SW,below=1cm of e1] (msb) {$\mathtt{msb}$};	

		\node[XOR] (x0) at (msb -| m1) {$\bigoplus$};

		\node[below=4cm of m1] (c1) {$C[m]$};
	
%
%		% \draw[edge] let \p1 = (m1),\p2 = (e1) in (m1) -- (\x1,\y2)  -- (e1);
		\draw[->] (e1) -- (msb);	
		\draw (msb) -- (x0);
		\draw (m1) -- (x0);
		\draw (x0) -- (c1);
		\draw[->] (zero) -- (e1);
		
	\end{scope}
		\end{tikzpicture}
		
&
		
\begin{tikzpicture}[
				scale=1,
				node distance=1.8cm,
				line width = 0.5pt,
			]
	
	\begin{scope}
		\node (m1) {$\Sigma$};
		\node[GW,below=1cm of m1] (e1) {$\tE^{*,N,\ell,b_1,b_2}_{K}$};	
		\node[below=3cm of m1] (c1) {$T$};
		
%
%		% \draw[edge] let \p1 = (m1),\p2 = (e1) in (m1) -- (\x1,\y2)  -- (e1);
		\draw[->] (m1) -- (e1);
		\draw (e1) -- (c1);
		
	\end{scope}
		\end{tikzpicture}		
		
\\

\scriptsize{ $\begin{aligned}
\Sigma = M[2] \oplus \ldots &\oplus M[m-2] \\&\oplus Z \oplus \underline{C[m]}\end{aligned}
$} & 
\scriptsize{
$\begin{aligned}
	\Sigma = M[2] \oplus \ldots &\oplus M[m-1] \\ &\oplus \underline{M[m]}
\end{aligned}
$% $\Sigma = M[2] \oplus \ldots \oplus M[m-1] \oplus \underline{M[m]}$
}

\end{tabular}
	


				\end{center}		
	\end{frame}


	\begin{frame}
		\frametitle{OTR's security}
		\begin{theorem}[{Theorem 3 of [Min14]}]
			If $\tE$ is a tweakable PRP, OTR is CPA-secure (privacy) and INT-CTXT-secure (authenticity).
		\end{theorem}

	\end{frame}

	\begin{frame}
		\frametitle{OTR's security: Privacy}
				\begin{center}
					\begin{tikzpicture}[
						scale=1,
						node distance=1.8cm,
						line width = 0.5pt,
					]
						%!TEX root = ../main.tex

\tikzstyle{GW}=[draw,rectangle,minimum height=2em,inner sep=3pt,thin]
\tikzstyle{SW}=[draw,rectangle,minimum height=1.25em,inner sep=1pt,thin]
\tikzstyle{XOR}=[inner sep=0pt]

	% \begin{scope}
	% 	\node (N) {$N$};
	% 	\node[SW,below=0.4cm of N] (p) {pad};
	% 	\node[GW,below = 0.4cm of p] (e) {$E_{K}$};
	% 	\node[below = 0.4cm of e] (delta) {$\delta$};
	%   		\node[SW, below = 0.4cm of delta] (mult) {$\times x^2$};
	%   		\node[ below = 0.4cm of mult] (L) {$L$};
	%
	% 	\draw[->] (N) -- (p);
	% 	\draw[->] (p) -- (e);
	% 	\draw[->] (e) -- (delta);
	% 	\draw[->] (delta) -- (mult);
	% 	\draw[->] (mult) -- (L);
	%
	% \end{scope}

	\begin{scope}
		\node (m1) {$M[1]$};
		\node[ right=3cm of m1.west] (m2) {$M[2]$};

		\node[GW,below right=0.5cm and 1cm of m1.south] (e1) {$\tE^{N,1,0}_{K}$};	
		\node[XOR] at (e1 -| m2) (x1) {$\bigoplus$};
		\node[GW,below=1.3cm of e1] (e2) {$\tE^{N,1,1}_{K}$};	
		\node[XOR] at (e2 -| m2) (x2) {$\bigoplus$};
		\node[below=4cm of m1] (c1) {$C[1]$};
		\node[ below=4cm of m2] (c2) {$C[2]$};

		\draw[->]  (m1) |- (e1);
		\draw (m2) -- (x1);  	
		\draw (e1) -- (x1);
%
%		\draw (e2 -| m2) -- (m1 |- e1);
		\draw let \p3 = (e2 -| m2) in (m1 |- e1) -- ++(0,-\crossoffset) -- (\x3,\y3+\crossoffset)  -- (x2);
%		
		\draw let \p1 = (m1 |- e2),\p2 = (e2),\p3 = (x1) in (x1) -- ++(0,-\crossoffset) -- (\x1,\y1+\crossoffset)  -- (c1);
%  			
		\draw[->] (m1 |- e2) -- (e2);
		\draw (e2) -- (x2);
%
 		\draw (x2) -- (c2);  	

	\end{scope}
  
					\end{tikzpicture}
				\end{center}		
	\end{frame}
	
% section otr_presentation (end)

\section{OTRv2's flaw} % (fold)
\label{sec:otrv2_s_flaw}


	\begin{frame}
		\frametitle{Instantiating the TBC}

		\begin{alertblock}{Remark}
			We are working in $\F_{2^n}$ represented as $\F_2[X]/P(X)$ with $P$ is a degree $n$ primitive polynomial in $\F_2$.
		\end{alertblock}

	    \begin{columns}
	     \column{.8\textwidth}
	\begin{itemize}
		\item Use the XE construction
	
		\item In {[Rog04]}: $\tE^{N,i,j}_{K}(M) = E_K(M + X^i (X+1)^j \delta)$ with $\delta = E_K(N)$
		
		
	\end{itemize}
	     \column{.2\textwidth}
					\begin{tikzpicture}[
						scale=1,
						node distance=1.8cm,
						line width = 0.5pt,
					]
						%!TEX root = ../main.tex

\tikzstyle{GW}=[draw,rectangle,minimum height=2em,inner sep=3pt,thin]
\tikzstyle{SW}=[draw,rectangle,minimum height=1.25em,inner sep=1pt,thin]
\tikzstyle{XOR}=[inner sep=0pt]

	\begin{scope}
		\node (m) {$M$};
		\node[XOR, below=0.5cm of m] (xor){$\bigoplus$};
		
		\node[XOR, left=0.5cm of xor] (delta) {$\Delta^N_{i,j}$};

		\node[GW, below=0.5cm of xor] (e){$E_K$};
		\node[XOR, below=0.5cm of e] (c){C};


		\draw  (m) -- (xor);
		\draw (delta) -- (xor);  	
		\draw[->] (xor) -- (e);
		\draw (e) -- (c);
	\end{scope}
  
					\end{tikzpicture}
		\end{columns}
	\end{frame}

	\begin{frame}
		\frametitle{Instantiating the TBC}

		\begin{alertblock}{Remark}
			We are working in $\F_{2^n}$ represented as $\F_2[X]/P(X)$ with $P$ is a degree $n$ primitive polynomial in $\F_2$.
		\end{alertblock}

	    \begin{columns}
	     \column{.8\textwidth}
		 
		 In OTRv1-v2 {[Min14]}, for efficiency, an other masking scheme is used:
		 % \begin{align*}
		 %    		 	\tE^{N,i,b}_{K}(M) &= E_K(M + (X^{i+1} + b) \delta) \\
		 %    		 	\tE^{*,N,i,b_1,b_2}_{K}(M) &= E_K(M + [(X+1)X^{\ell+1} + X\cdot b_1 +b_1 +b_2]\delta) \\
		 % \end{align*}
		 \begin{align*}
   		 	\Delta^{N}_{i,b} &= (X^{i+1} + b) \delta \\
   		 	\Delta^{*,N}_{\ell,b_1,b_2} &= [(X+1)X^{\ell+1} + X\cdot b_1 +b_1 +b_2]\delta \\
		 \end{align*}

	     \column{.2\textwidth}
					\begin{tikzpicture}[
						scale=1,
						node distance=1.8cm,
						line width = 0.5pt,
					]
						%!TEX root = ../main.tex

\tikzstyle{GW}=[draw,rectangle,minimum height=2em,inner sep=3pt,thin]
\tikzstyle{SW}=[draw,rectangle,minimum height=1.25em,inner sep=1pt,thin]
\tikzstyle{XOR}=[inner sep=0pt]

	\begin{scope}
		\node (m) {$M$};
		\node[XOR, below=0.5cm of m] (xor){$\bigoplus$};
		
		\node[XOR, left=0.5cm of xor] (delta) {$\Delta^N_{i,j}$};

		\node[GW, below=0.5cm of xor] (e){$E_K$};
		\node[XOR, below=0.5cm of e] (c){C};


		\draw  (m) -- (xor);
		\draw (delta) -- (xor);  	
		\draw[->] (xor) -- (e);
		\draw (e) -- (c);
	\end{scope}
  
					\end{tikzpicture}
		\end{columns}
	\end{frame}

	\begin{frame}
		\frametitle{The flaw}

		\begin{lemma}[{Lemma 1 of [Min14]}]
			The TBC is indistinguishable from a tweakable PRP.
		\end{lemma}
		The proof of this lemma relies on the following claim
		\begin{block}{Claim}
		\begin{align*}
		\mbox{Let }\mathcal{S}_1(\delta) = & \left\{X^{i+1}\delta, (X^{i+1} +1)\delta,\right\} \\
		 &\tab \cup \left\{ (X^{i+2} + X^{i+1} + b_1 X + b_2)\delta \right\}_{i=1, b_1 \in \zeon, b_2 \in \zeon} 
		\end{align*}
		The elements of $\mathcal{S}_1(\delta)$
		are pairwise different.
		\end{block}
	
		\begin{alertblock}{Our attack}<2->
			This is not true in general!
		\end{alertblock}

	\end{frame}

	\begin{frame}
		\frametitle{The trick}

		\begin{itemize}
			\item In [Rog04], for $\log_X (X+1) = \alpha$, as long as $0 \leq i + \alpha j \leq 2^{n}-1$, $\{X^i (X+1)^j\}$ are pairwise distinct 
			\\ \tab $\Rightarrow$ bound $i$ and $j$.
			
			\item<2-> In [Min14], we cannot show that, for some $q$ elements are pairwise distinct in 
				\[
				\left\{X^{i+1}, X^{i+1} +1 \right\} \cup \left\{ X^{i+2} + X^{i+1} + b_1 X + b_2\right\}_{1\leq i \leq q, (b_1,b_2) \in \zeon^2}.
				\]
		
			\item<3-> If $P(X) = X^n + X^j + 1$, there is a collision between $X^n$ and $X^j + 1$ in $\F_{2^n} = \F_2[X]/P(X)$.
		\end{itemize}
	\end{frame}

	\begin{frame}
		\frametitle{For actual block sizes ($n = 64, 128$)?}

		\begin{itemize}
			\item If $8 | n$, $\F_{2^n} = \F_2[X]/P(X)$ with $P$ with at least 5 non-zero coefficient ($P(X) = X^n + X^{j_1} + X^{j_2} + X^{j_3} +1$).
			\\ \tab $\Rightarrow$ no immediate collisions in general.
			
			\item<2-> For SW/HW efficiency, we usually choose $P$ such that its non-zero coefficients are close to each other, preferably in the least significant bytes.
			\begin{align*}
				P_{64}(X) &= X^{64}+X^4+X^3+X+1 \\
				P_{128}(X) &= X^{128}+X^{7}+X^{2}+X+1
			\end{align*}
			
			\item<3-> For $n = 64$ with the usual $P$, we have a collision of the type $X^i = X^{j+1} + X^j + X + 1$ :
			 \[
			 X^{64} = X^4+X^3+X+1
			 \]
		\end{itemize}
		
	\end{frame}

% section otrv2_s_flaw (end)


\section{Consequences} % (fold)
\label{sec:consequences}

	\begin{frame}
		\frametitle{Consequences}

		\begin{alertblock}{Problem}
There is a flaw in the proof of OTR, even for practical parameters.
		\end{alertblock}

\vspace{0.4cm}

Does the confidentiality of OTR break?

Does the unforgeability of OTR break?


		% \begin{itemize}
		% 	\item There is a flaw in the proof, even for practical parameters.
		%
		% 	\item Does the confidentiality of OTR fall?
		%
		% 	\item Does the unforgeability of OTR fall?
		%
		% \end{itemize}
		
	\end{frame}

	\begin{frame}
		\frametitle{Typology of collisions}

		We can encounter three types of collision among the tweaks' polynomials:
		\begin{align}
			X^i &= X^j + 1 \\
			X^i & = X^{j+1} + X^j + r(x) \\
			X^{i+1} + X^i & = X^{j+1} + X^j + r(x)
		\end{align}
		with $r(X) \in \{0, 1, X, X+1\}$.
	
	\end{frame}

	\begin{frame}
		\frametitle{Attacks}

		\begin{block}{Out attack}
		\begin{description}
			\item[Type 1 ($X^i = X^j + 1$)] \ \\
			Break confidentiality \emph{and} unforgeability.   
			\item[Type 2 ($X^i  = X^{j+1} + X^j + r(x)$)] \ \\ Break confidentiality if $i < j$. Break unforgeability o/w.   
			\item[Type 3 ($X^{i+1} + X^i  = X^{j+1} + X^j + r(x)$)]  \ \\
			Break unforgeability.   
		\end{description}
		\end{block}

		Idea: use the collision to have relations between block cipher's inputs and create collisions on the outputs.
		
		Only \emph{one} query to the encryption oracle, with a message of $\max(i,j)$ blocks.
	\end{frame}


% section consequences (end)

\section{Practical security for $n = 128$.} % (fold)
\label{sec:practical_security_for_n_128}


	\begin{frame}
		\frametitle{$n = 128$ in practice}

		Usually, for $n = 128$, we choose
		\[
			P(X) = X^{128}+X^{7}+X^{2}+X+1.
		\]
		There is no trivial collision. 
		\begin{block}{Remark}
			This is not true for all irreducible $P$ of degree $128$.\\
			Ex: $P(X)=X^{128}+X^{127}+X^{61}+X^{60}+1$
		\end{block}
		
		Can we find a collision among tweaks polynomial?
	\end{frame}

	\begin{frame}
		\frametitle{In search for lost collision}

		\begin{itemize}
			\item We are only interested in collisions with $i$ and $j < 2^{64}$: the security proof of OTR only holds up to the birthday bound.
			
			\item<2-> We cannot find such collisions by constructing a collision in $\F_{2^{64}}$ and then `moving' it to $\F_{2^{128}}$.
			
			\item<3-> Our only hope: exhaustive search. 
			
			\item<4-> Generate, sort and match tweak polynomials (Embarrassingly parallelizable).
			
			\item<5-> Problem: requires $O(n 2^n)$ memory and $O(n 2^n)$ time ...
		\end{itemize}
	\end{frame}

	\begin{frame}
		\frametitle{In search for lost collision}



We used time/memory tradeoffs to search for any collision with $i, j < 2^{45}$.

		\begin{theorem}
			There is no collision among the tweaks polynomials for $i, j < 2^{45}$ when $F_{2^{128}}$ is defined as $F_2[X]/X^{128}+X^{7}+X^{2}+X+1$.
		\end{theorem}
		
		The exhaustive search took 15 CPU-years using 3TB of RAM.


		\begin{exampleblock}{Question}<2->
			What about $2^{45} \geq i, j$ ?
		\end{exampleblock}
	\end{frame}

	\begin{frame}
		\frametitle{Probable collision before the birthday bound}

		\begin{itemize}
			\item If tweak polynomials behaved like random polynomials, we should have a collision just before the birthday bound.
			
			\item For $n = 32, 64$, we enumerated the irreducible polynomials over $\F_2$ of degree $n$ and search for the lowest degree colliding polynomials.
		\end{itemize}
		
	\end{frame}



	\begin{frame}
		\frametitle{First collision for $n=32$}
		
		\centering
		\scalebox{0.8}{
			\begin{tikzpicture}[gnuplot]
%% generated with GNUPLOT 5.0p3 (Lua 5.2; terminal rev. 99, script rev. 100)
%% Ven 19 aoû 20:01:38 2016
\path (0.000,0.000) rectangle (13.000,6.000);
\gpcolor{color=gp lt color border}
\node[gp node right,rotate=30] at (0.803,0.982) {$4 < d \leq 5$};
\node[gp node right,rotate=30] at (1.947,0.982) {$5 < d \leq 6$};
\node[gp node right,rotate=30] at (3.092,0.982) {$6 < d \leq 7$};
\node[gp node right,rotate=30] at (4.236,0.982) {$7 < d \leq 8$};
\node[gp node right,rotate=30] at (5.380,0.982) {$11 < d \leq 12$};
\node[gp node right,rotate=30] at (6.524,0.982) {$12 < d \leq 13$};
\node[gp node right,rotate=30] at (7.668,0.982) {$13 < d \leq 14$};
\node[gp node right,rotate=30] at (8.812,0.982) {$14 < d \leq 15$};
\node[gp node right,rotate=30] at (9.957,0.982) {$15 < d \leq 16$};
\node[gp node right,rotate=30] at (11.101,0.982) {$16 < d \leq 17$};
\gpsetlinetype{gp lt border}
\gpsetdashtype{gp dt solid}
\gpsetlinewidth{1.00}
\draw[gp path] (0.460,1.320)--(11.444,1.320);
\gpfill{rgb color={0.678,0.847,0.902}} (0.517,1.320)--(1.090,1.320)--(1.090,1.608)--(0.517,1.608)--cycle;
\gpcolor{rgb color={0.000,0.000,0.000}}
\draw[gp path] (0.517,1.320)--(0.517,1.607)--(1.089,1.607)--(1.089,1.320)--cycle;
\gpfill{rgb color={0.678,0.847,0.902}} (1.661,1.320)--(2.234,1.320)--(2.234,1.378)--(1.661,1.378)--cycle;
\draw[gp path] (1.661,1.320)--(1.661,1.377)--(2.233,1.377)--(2.233,1.320)--cycle;
\gpfill{rgb color={0.678,0.847,0.902}} (5.094,1.320)--(5.667,1.320)--(5.667,1.436)--(5.094,1.436)--cycle;
\draw[gp path] (5.094,1.320)--(5.094,1.435)--(5.666,1.435)--(5.666,1.320)--cycle;
\gpfill{rgb color={0.678,0.847,0.902}} (6.238,1.320)--(6.811,1.320)--(6.811,1.551)--(6.238,1.551)--cycle;
\draw[gp path] (6.238,1.320)--(6.238,1.550)--(6.810,1.550)--(6.810,1.320)--cycle;
\gpfill{rgb color={0.678,0.847,0.902}} (7.382,1.320)--(7.955,1.320)--(7.955,2.815)--(7.382,2.815)--cycle;
\draw[gp path] (7.382,1.320)--(7.382,2.814)--(7.954,2.814)--(7.954,1.320)--cycle;
\gpfill{rgb color={0.678,0.847,0.902}} (8.526,1.320)--(9.099,1.320)--(9.099,5.345)--(8.526,5.345)--cycle;
\draw[gp path] (8.526,1.320)--(8.526,5.344)--(9.098,5.344)--(9.098,1.320)--cycle;
\gpfill{rgb color={0.678,0.847,0.902}} (9.671,1.320)--(10.244,1.320)--(10.244,4.684)--(9.671,4.684)--cycle;
\draw[gp path] (9.671,1.320)--(9.671,4.683)--(10.243,4.683)--(10.243,1.320)--cycle;
\gpfill{rgb color={0.678,0.847,0.902}} (10.815,1.320)--(11.388,1.320)--(11.388,1.838)--(10.815,1.838)--cycle;
\draw[gp path] (10.815,1.320)--(10.815,1.837)--(11.387,1.837)--(11.387,1.320)--cycle;
\gpcolor{color=gp lt color border}
\node[gp node center] at (0.803,1.895) {10};
\node[gp node center] at (1.947,1.665) {2};
\node[gp node center] at (3.092,1.607) {0};
\node[gp node center] at (4.236,1.607) {0};
\node[gp node center] at (5.380,1.722) {4};
\node[gp node center] at (6.524,1.837) {8};
\node[gp node center] at (7.668,3.102) {52};
\node[gp node center] at (8.812,5.631) {140};
\node[gp node center] at (9.957,4.970) {117};
\node[gp node center] at (11.101,2.125) {18};
\draw[gp path] (0.460,1.320)--(11.444,1.320);
%% coordinates of the plot area
\gpdefrectangularnode{gp plot 1}{\pgfpoint{0.460cm}{1.320cm}}{\pgfpoint{11.444cm}{5.631cm}}
\end{tikzpicture}
%% gnuplot variables

			}
	\end{frame}
	\begin{frame}
		\frametitle{First collision for $n=64$}
		
		\centering
		\scalebox{0.8}{
			\begin{tikzpicture}[gnuplot]
%% generated with GNUPLOT 5.0p3 (Lua 5.2; terminal rev. 99, script rev. 100)
%% Ven 19 aoû 20:01:38 2016
\path (0.000,0.000) rectangle (12.500,6.000);
\gpcolor{color=gp lt color border}
\node[gp node right,rotate=30] at (0.827,0.982) {$5 < d \leq 6$};
\node[gp node right,rotate=30] at (2.049,0.982) {$6 < d \leq 26$};
\node[gp node right,rotate=30] at (3.271,0.982) {$26 < d \leq 27$};
\node[gp node right,rotate=30] at (4.493,0.982) {$27 < d \leq 28$};
\node[gp node right,rotate=30] at (5.715,0.982) {$28 < d \leq 29$};
\node[gp node right,rotate=30] at (6.938,0.982) {$29 < d \leq 30$};
\node[gp node right,rotate=30] at (8.160,0.982) {$30 < d \leq 31$};
\node[gp node right,rotate=30] at (9.382,0.982) {$31 < d \leq 32$};
\node[gp node right,rotate=30] at (10.604,0.982) {$32 < d \leq 33$};
\gpsetlinetype{gp lt border}
\gpsetdashtype{gp dt solid}
\gpsetlinewidth{1.00}
\draw[gp path] (0.460,1.320)--(10.971,1.320);
\gpfill{rgb color={0.678,0.847,0.902}} (0.521,1.320)--(1.133,1.320)--(1.133,1.363)--(0.521,1.363)--cycle;
\gpcolor{rgb color={0.000,0.000,0.000}}
\draw[gp path] (0.521,1.320)--(0.521,1.362)--(1.132,1.362)--(1.132,1.320)--cycle;
\gpfill{rgb color={0.678,0.847,0.902}} (2.966,1.320)--(3.578,1.320)--(3.578,1.349)--(2.966,1.349)--cycle;
\draw[gp path] (2.966,1.320)--(2.966,1.348)--(3.577,1.348)--(3.577,1.320)--cycle;
\gpfill{rgb color={0.678,0.847,0.902}} (4.188,1.320)--(4.800,1.320)--(4.800,1.391)--(4.188,1.391)--cycle;
\draw[gp path] (4.188,1.320)--(4.188,1.390)--(4.799,1.390)--(4.799,1.320)--cycle;
\gpfill{rgb color={0.678,0.847,0.902}} (5.410,1.320)--(6.022,1.320)--(6.022,1.655)--(5.410,1.655)--cycle;
\draw[gp path] (5.410,1.320)--(5.410,1.654)--(6.021,1.654)--(6.021,1.320)--cycle;
\gpfill{rgb color={0.678,0.847,0.902}} (6.632,1.320)--(7.244,1.320)--(7.244,2.461)--(6.632,2.461)--cycle;
\draw[gp path] (6.632,1.320)--(6.632,2.460)--(7.243,2.460)--(7.243,1.320)--cycle;
\gpfill{rgb color={0.678,0.847,0.902}} (7.854,1.320)--(8.466,1.320)--(8.466,5.048)--(7.854,5.048)--cycle;
\draw[gp path] (7.854,1.320)--(7.854,5.047)--(8.465,5.047)--(8.465,1.320)--cycle;
\gpfill{rgb color={0.678,0.847,0.902}} (9.077,1.320)--(9.689,1.320)--(9.689,5.298)--(9.077,5.298)--cycle;
\draw[gp path] (9.077,1.320)--(9.077,5.297)--(9.688,5.297)--(9.688,1.320)--cycle;
\gpfill{rgb color={0.678,0.847,0.902}} (10.299,1.320)--(10.911,1.320)--(10.911,1.641)--(10.299,1.641)--cycle;
\draw[gp path] (10.299,1.320)--(10.299,1.640)--(10.910,1.640)--(10.910,1.320)--cycle;
\gpcolor{color=gp lt color border}
\node[gp node center] at (0.827,1.570) {6};
\node[gp node center] at (2.049,1.529) {0};
\node[gp node center] at (3.271,1.556) {4};
\node[gp node center] at (4.493,1.598) {10};
\node[gp node center] at (5.715,1.862) {48};
\node[gp node center] at (6.938,2.669) {164};
\node[gp node center] at (8.160,5.256) {536};
\node[gp node center] at (9.382,5.506) {572};
\node[gp node center] at (10.604,1.848) {46};
\draw[gp path] (0.460,1.320)--(10.971,1.320);
%% coordinates of the plot area
\gpdefrectangularnode{gp plot 1}{\pgfpoint{0.460cm}{1.320cm}}{\pgfpoint{10.971cm}{5.631cm}}
\end{tikzpicture}
%% gnuplot variables

			}
	\end{frame}


	\begin{frame}
		\frametitle{Conjecture for $n=128$}
		
		\begin{block}{Conjecture}
			There is no collision among the tweaks polynomials for $i, j < 2^{60}$ when $F_{2^{128}}$ is defined as $F_2[X]/X^{128}+X^{7}+X^{2}+X+1$.
		\end{block}
	\end{frame}

% section practical_security_for_n_128 (end)

\section*{Conclusion} % (fold)
\label{sec:conclusion}

	\begin{frame}
		\frametitle{Conclusion}
		
		\begin{itemize}
			\item OTRv2 is insecure for many block sizes.
			\item OTRv2 is for $n = 128$ when the message length is limited to $2^{45}$ blocks.
			\item OTRv2 is probably secure for $n = 128$ almost up to the birthday bound.
			\item OTRv3 fixes the issue.
		\end{itemize}
	\end{frame}


	\begin{frame}

		\begin{center}
			{\Large Thank you!}
			
			\vspace{1cm}
			Paper: \href{http://ia.cr/2016/234}{ia.cr/2016/234}
		\end{center}
	\end{frame}
	
% section conclusion (end)


\end{document}